\section{Logistic equation}

\subsection{Why Logistic Equation}

\textbf{Logistic equation} is an \textit{Ordinary Differential Equation} that is
used to analyze the population change in a system. Before the theory of logistic
model is established, people were considering population using exponential
model (in which the rate of growth/decay is directly proportional to the current
population), and the differential equation that describes this is:

$$
\frac{dN}{dt}=rN
\footnote{$N$ stands for the current population, $t$ stands for time, and $r$
is the speed factor of the rate of change}
$$

In fact, a system may also have a limitation of resources for a species to live,
so the proposer Malthus believed that the population in a system may oscillate
and eventually will reach a number, which is known as the \textbf{Carrying
capacity}. Therefore, the differential equation of logistic model is:

$$
\frac{dN}{dt}=rN\left(\frac{K-N}K\right)
\footnote{$K$ is the carrying capacity}
$$

As you can see, when $N$ is greater than $K$, the population will decrease, and
if $N$ is less than $K$, the population will increase. Since the size of $N$
affects the sign of the derivative.

\subsection{Solve Logistic Equation}

Since we have known the origin and the meaning of this differential equation,
we can now figure a function out that can satisfy this population model along
with the equation.

Using Leibniz notation to represent derivatives can help us solve differential
equation by separating the differentiated variables:

$$dN=rN\left(\frac{K-N}k\right)dt$$

Since $r$ and $K$ are constants, we can leave $r$ alone and let the whole
polynomial related to $N$ be on the left side of the equation:

$$
\begin{aligned}
	\frac{dN}{N(K-N)}=\frac{r}Kdt \\
	\int\frac{dN}{N(K-N)}=\int\frac{r}Kdt
\end{aligned}
$$

The right part we can immediately come up with the answer $\frac{r}Kt+C$, but
the left part we need to cope with it longer:

$$
\begin{aligned}
	\because\int\frac{dN}{N(K-N)}=\int\frac{A}{N}dN+\frac{B}{K-N}dN \\
	\therefore A(K-N)+BN=1 \\
	\therefore AK-AN+BN=1 \\
	\therefore AK-N(A-B)=1 \\
\end{aligned}
$$

Since we know that $N$ is a variable and $K$ is a constant, we can quickly
conclude that the factor of the variable must be zero in order to let the result
be $1$:

$$
\begin{aligned}
	\because\begin{cases}
		AK=1 \\
		A-B=0
	\end{cases} \\
	\therefore A=B=\frac1K
\end{aligned}
$$

Therefore the integral of the left part is:

$$
\begin{aligned}
	\int\frac{dN}{N(K-N)}
	&=\frac1K\left(\int\frac{dN}{N}+\int\frac{dN}{K-N}\right) \\
	&=\frac1K\left(\ln(N)-\ln(K-N)+C_0\right) \\
	&=\frac1K\ln(\frac{N}{K-N})+C'
\end{aligned}
$$

Because both left and right parts of the equations are integrated, we can now
evaluate them to the original equation:

$$
\begin{aligned}
	\frac1K\ln(\frac{N}{K-N})+C'=\frac{r}Kt+C \\
	\ln(\frac{N}{K-N})=rt+C'' \\
	\frac{N}{K-N}=C_0e^{rt} \\
	N=C_0e^{rt}(K-N) \\
	N(1+C_0e^{rt})=KC_0e^{rt} \\
	N=\frac{KC_0e^{rt}}{1+C_0e^{rt}} \\
	N=N(t)=\frac{K}{1+C_1e^{-rt}}
\end{aligned}
$$

Finally we obtain the solution of this equation, which is called logistic
function. However this is not finished yet, the meaning of $C_1$ is not clear
enough. We can add a limiting condition where $N(0)=N_0$\footnote{To specify
the initial population of the system}:

$$
N(0)=\frac{K}{1+C_1e^{-r0}}=\frac{K}{1+C_1}
$$

We know that when $t=0$, $N$ shall be equal to the initial population $N_0$, so:

$$
\begin{aligned}
	N_0=\frac{K}{1+C_1} \\
	N_0(1+C_1)=K \\
	N_0C_1=K-N_0 \\
	C_1=\frac{K-N_0}{N_0}
\end{aligned}
$$

Let's plug $C_1$ in terms of $N_0$ to the function:

$$
\begin{aligned}
	N(t)
	&=\frac{K}{1+\frac{K-N_0}{N_0}e^{-rt}} \\
	&=\frac{KN_0}{N_0+(K-N_0)e^{-rt}}
\end{aligned}
$$

Finally we have the formula for logistic function:

$$
N(t)=\frac{KN_0}{N_0+(K-N_0)e^{-rt}}
$$
